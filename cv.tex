\documentclass[notitle,nologo,booktabs]{europecv}
\usepackage[a4paper,margin=1in]{geometry}
\usepackage{hyperref}

% personal data
\ecvname{Podsztavek, Ondřej}
\ecvdateofbirth{May 19, 1995}
%\ecvemail{ondrej.podsztavek@gmail.com}
\ecvemail{podszond@fit.cvut.cz}
\ecvaddress{Houdkovice 11\\& Trnov 517 33\\& Czech Republic}
\ecvtelephone{+420 604 236 179}

\begin{document}
\begin{europecv}
	\ecvpersonalinfo
	%\ecvitem{Website}{\url{https://podondra.cz}}
	\ecvitem{GitHub}{\url{https://github.com/podondra}}

	\ecvsection{Education}
	\ecvitem{2017--Present}{
		\textbf{Master's degree}\\&
		Czech Technical University in Prague, Czech Republic\\&
		Branch of study: Knowledge Engineering
	}
	\ecvitem{2014--2017}{
		\textbf{Bachelor's degree}\\&
		Czech Technical University in Prague, Czech Republic\\&
		Branch of study: Computer Science\\&
		\textit{Graduated with honours}\\&
		Thesis: Deep Learning in Large Astronomical Spectra Archives\\&
		Supervisor: Dr. Petr Škoda, Ph.D.\\&
		\textit{My bachelor's thesis aimed to find emission-line
		spectra using deep learning. Firstly,
		I trained and compared two neural networks
		(dense and VGG-like convolutional network)
		on spectral data from Ondřejov telescope.
		Then I used the convolutional network to classify
		spectra from a huge LAMOST archive.
		The result showed that the network is able to
		discover yet unknown stars with interesting physical
		properties.}
	}

	\ecvsection{Research experience}
	\ecvitem{Nov 2016--Dec 2017}{
		\textbf{Researcher}\\&
		Astronomical Institute of the Czech Academy of Sciences,\\&
		Ondřejov, Czech Republic\\&
		Application of Neural Networks to Stellar Spectra\\&
		\textit{My bachelor's thesis pointed out benefits of using
		Machine Learning in Astronomy.
		Therefore, I started to work
		on an application of different neural network types
		to stellar spectra which also included a great amount of
		work in preprocessing and data balancing domains.}
	}

	\ecvsection{Community involvement}
	\ecvitem{Let's talk ML Prague}{
		Let's Talk ML is once a fortnight meeting of students interested
		in Machine Learning and Artificial Intelligence.
		The format is usually two short talks followed by discussion.
		I am a regular visitor and I already had two talks about
		\textit{Transfer Learning} and \textit{Deep Q-Network}.
	}

	\ecvsection{Oral presentations}
	\ecvitem{Mar 2018}{
		\textbf{Reinforcement Learning in Recommendation:
		Off-policy Policy Evaluation}.
		My presentation at the meeting of
		Machine Learning and Computational Intelligence Group (ML-CIG),
		Prague.\\&
		\textit{Presentation which addressed the evaluation problem
		of new policy from off-line data in order to ascertain
		that it can be safely deployed.
		I showed the problem from point of view of both contextual
		bandits and full reinforcement learning formalization.
		I presented different algorithms for off-policy policy
		evaluation as direct method, inverse propensity score,
		double robust estimator, important sampling and other
		modification of these methods.}
	}
	\ecvitem{Jun 2017}{
		\textbf{Deep Learning in Large Astronomical Archives}.
		Bachelor's thesis presented at Symposium 14 - Astroinformatics:
		From big data to understanding the universe at large,
		European Week of Astronomy and Space Science (EWASS 2017),
		Prague.\\&
		\textit{Talk about my bachelor's thesis presented at the
		largest
		European astronomical conference. I focused the talk
		on domain adaptation and learning from imbalanced data.
		Firstly, I emphasized the problem of domain adaptation,
		which using data from a different telescope tries to
		train a machine learning algorithm to classify spectra from
		a totally different telescope.
		Secondly, I talk about the challenge of imbalanced learning
		because interesting objects are usually in minority therefore
		are difficult to learn about and identify.
		I proposed SMOTE as a solution which I experimentally verified.}
	}

	%\ecvsection{Work experience}
	%\ecvitem{Dec 2015--Jan 2017}{
	%	\textbf{Solaris Junior Engineer}\\&
	%	Oracle, Prague (Czech Republic)\\&
	%	Oracle Solaris Globalization and Information Engineering
	%	development
	%}

	\ecvsection{Specialized Courses}
	\ecvitem{Nov 2017}{
		\textbf{From Complexity to Intelligence}\\&
		Télécom ParisTech, Paris, France\\&
		\textit{A one-week intensive course concerned with connecting
		Artificial Intelligence with Kolmogorov complexity and
		randomness.
		I was introduced to the mathematical notion of complexity
		which was then used to study reasoning, perception and
		decision making.}
	}

	\ecvsection{Personal skills}
	\ecvitem{Programming languages}{Python, C, C++}
	\ecvitem{Scientific libraries}{
		TensorFlow, scikit-learn, NumPy, Pandas
	}

	\ecvsection{Languages}
	\ecvitem{English}{
		CEFR level: \textit{C1 (Proficient user)}\\&
		Obtained by International English Language Testing System\\&
		Test Report Form: Academic
	}

	\ecvsection{References}
	\ecvitem{Dr. Petr Škoda, Ph.D.}{
		Senior Researcher in Astroinformatics\\&
		Stellar Physics Department\\&
		Astronomical Institute of the Czech Academy of Sciences\\&
		skoda@sunstel.asu.cas.cz
	}
\end{europecv}
\end{document}
