\documentclass[totpages,nologo,booktabs]{europecv}
\usepackage{graphicx}
\usepackage[a4paper,margin=1in]{geometry}
\usepackage{hyperref}

% personal data
\ecvname{Podsztavek, Ondřej}
\ecvdateofbirth{May 19, 1995}
%\ecvemail{\href{mailto:ondrej.podsztavek@gmail.cz}{ondrej.podsztavek@gmail.cz}}
\ecvprofessional{\href{mailto:podszond@fit.cvut.cz}{podszond@fit.cvut.cz}}
\ecvaddress{Houdkovice 11, Trnov 517 33, Czech Republic}
%\ecvtelephone{+420 604 236 179}
\ecvnationality{Czech}

\begin{document}
\begin{europecv}
	\ecvpersonalinfo
	%\ecvitem{Website}{\url{https://podondra.cz}}
	\ecvitem{GitHub}{\url{https://github.com/podondra}}

	\ecvsection{Research experience}
	\ecvitem{Jun 2018--Present}{
		\textbf{Junior Researcher}\\&
		\textit{Faculty of Information Technology, CTU in Prague}\\&
		I research applications of machine learning algorithms to
		Big Data in astronomical spectroscopy
		as a member of the High Performance Computing reasearch group
		in the Reasearch Center for Infromatics (RCI).

	}
	\ecvitem{Nov 2016--Dec 2017}{
		\textbf{Researcher}\\&
		\textit{Astronomical Institute of the Czech Academy of Sciences}\\&
		\textit{Application of Neural Networks to Stellar Spectra}\\&
		My bachelor's thesis pointed out benefits of using
		Machine Learning in Astronomy.
		Therefore, I started to work
		on an application of different neural network types
		to stellar spectra which also included a great amount of
		work in preprocessing and data balancing domains.
	}

	\ecvsection{Education}
	\ecvitem{2017--Present}{
		\textbf{Master's degree}\\&
		\textit{Faculty of Information Technology, CTU in Prague}\\&
		Branch of study: \textit{Knowledge Engineering}\\&
		Thesis: \textit{Neural Network Based Domain Adaptation in Spectroscopis Sky Surveys}\\&
		%Supervisor: \textit{RNDr. Petr Škoda, CSc.}\\&
		The goal of the thesis is the analysis of the impact
		of domain adaptation in astronomical archives
		with a focus on neural networks that would allow
		using labelled data from one ground-based telescope
		or space mission archive to discover knowledge
		in another archive.
	}
	\ecvitem{2014--2017}{
		\textbf{Bachelor's degree}\\&
		\textit{Faculty of Information Technology, CTU in Prague}\\&
		Branch of study: \textit{Computer Science}\\&
		\textit{Graduated with honours}\\&
		Thesis: \textit{Deep Learning in Large Astronomical Spectra Archives}\\&
		%Supervisor: \textit{RNDr. Petr Škoda, CSc.}\\&
		My bachelor's thesis aimed to find emission-line
		spectra using deep learning. Firstly,
		I trained and compared two neural networks
		(dense and VGG-like convolutional network)
		on spectral data from Ondřejov telescope.
		Then I used the convolutional network to classify
		spectra from a huge LAMOST archive.
		The result showed that the network is able to
		discover yet unknown stars with interesting physical
		properties.
	}

	\ecvsection{Study abroad}
	\ecvitem{Aug 2018--Dec 2018}{
		\textbf{Study abroad}\\&
		\textit{Nanyang Technological University, Singapore}\\&
		I spent a semester at the Nanyang Technological
		University in Singapure, where I studied subject connected
		to my research interests: Computational Intelligence, Astronomy,
		Neural \& Fuzzy Systems, Computer Vision.
	}
	\ecvitem{Nov 2017}{
		\textbf{From Complexity to Intelligence}\\&
		\textit{Télécom ParisTech, Paris, France}\\&
		A one-week intensive course concerned with connecting
		Artificial Intelligence with Kolmogorov complexity and
		randomness.
		I was introduced to the mathematical notion of complexity
		which was then used to study reasoning, perception and
		decision making.
	}

	\ecvsection{Publications}
	\ecvitem{In preparation}{
		\textbf{An active deep learning method for discovery of objects of interest in large spectra surveys}\\&
		In: \textit{Astronomy \& Astrophysics}
		}
	\ecvitem{2018}{
		\textbf{Comparing Offline and Online Evaluation Results of Recommender Systems}\\&
		REVEAL Workshop paper\\&
		In: \textit{Proceedings of RecSyS conference (RecSyS’18)}
		}

	\ecvsection{Oral presentations}
	\ecvitem{Oct 2019}{
		\textbf{Anomaly Detection in Air Pollution Data}\\&
		\textit{DaZ \& WIKT 2019}\\&
		Presentation of my semestral work in the DaZ \& WIKR 2019
		conference.
		I used machine learning techniques
		as linear regression and LSTM recurrent network
		to detect anomalies in the air pollution data
		which were publish as open data by the city of Prague.
	}
	\ecvitem{Mar 2018}{
		\textbf{Reinforcement Learning in Recommendation:
		Off-policy Policy Evaluation}\\&
		\textit{Machine Learning and Computational Intelligence Group, Prague}\\&
		Presentation which addressed the evaluation problem
		of new policy from off-line data in order to ascertain
		that it can be safely deployed.
		I showed the problem from point of view of both contextual
		bandits and full reinforcement learning formalization.
		I presented different algorithms for off-policy policy
		evaluation as direct method, inverse propensity score,
		double robust estimator, important sampling and other
		modification of these methods.
	}
	\ecvitem{Jun 2017}{
		\textbf{Deep Learning in Large Astronomical Archives}.\\&
		\textit{Symposium 14 -- Astroinformatics, EWASS 2017}\\&
		Talk about my bachelor's thesis presented at the largest
		European astronomical conference. I focused the talk
		on domain adaptation and learning from imbalanced data.
		Firstly, I emphasized the problem of domain adaptation,
		which using data from a different telescope tries to
		train a machine learning algorithm to classify spectra from
		a totally different telescope.
		Secondly, I talk about the challenge of imbalanced learning
		because interesting objects are usually in minority therefore
		are difficult to learn about and identify.
		I proposed SMOTE as a solution which I experimentally verified.
	}

	\ecvsection{Work experience}
	\ecvitem{Dec 2015--Jan 2017}{
		\textbf{Solaris Junior Engineer}\\&
		\textit{Oracle, Prague (Czech Republic)}\\&
		Oracle Solaris globalization and information engineering
		development.
	}

	\ecvsection{Awards}
	\ecvitem{2019}{
		\textbf{Cena Stanislava Hanzla}\\&
		Annually, the award is given to a student from each faculty
		of the CTU in Prague.
		Prerequisites are excellent study results
		and involvement in a reaserch activity.
	}

	%\ecvsection{Personal skills}
	%\ecvitem{Programming languages}{Python, C, C++}
	%\ecvitem{Scientific libraries}{
	%	TensorFlow, scikit-learn, NumPy, Pandas
	%}

	\ecvsection{Languages}
	\ecvitem{English}{
		\textbf{International English Language Testing System (IELTS)}\\&
		CEFR level: \textit{C1 (Proficient user)}\\&
		Test Report Form: \textit{Academic}\\&
		Obtained in 2013.
	}

	%\ecvsection{Community involvement}
	%\ecvitem{Let's talk ML Prague}{
	%	Let's Talk ML is once a fortnight meeting of students interested
	%	in Machine Learning and Artificial Intelligence.
	%	The format is usually two short talks followed by discussion.
	%	I am a regular visitor and I already had two talks about
	%	\textit{Transfer Learning} and \textit{Deep Q-Network}.
	%}

	%\ecvsection{References}
	%\ecvitem{RNDr. Petr Škoda, CSc.}{
	%	Senior Researcher in Astroinformatics\\&
	%	Stellar Physics Department\\&
	%	Astronomical Institute of the Czech Academy of Sciences\\&
	%	\href{mailto:skoda@sunstel.asu.cas.cz}{skoda@sunstel.asu.cas.cz}
	%}
\end{europecv}
\end{document}
